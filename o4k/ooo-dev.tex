%\documentclass[handout]{beamer}
\documentclass{beamer}
\usepackage[utf8]{inputenc}
\usepackage[T1]{fontenc}
\usepackage[francais]{babel}
\usepackage{eurosym}

\usetheme[opacity=0.0]{diepen}

\title{Développement du module d'annotation dans OOo Impress}
\author{Clément~\textsc{Delafargue} \and Morgan~\textsc{Magnin} \and Nelle~\textsc{Varoquaux}\and Benjamin~\textsc{Vialle}}
\institute[\textsc{ECN}]{École Centrale de Nantes}
\date{11 juillet 2011}



%\useinnertheme[shadow=false]{rounded}
\setbeamerfont{block title}{size={},series=\bfseries}


\setbeamercolor*{frametitle}{fg=black}
\setbeamercolor*{title}{fg=black}
\setbeamercolor*{subtitle}{fg=black}
\setbeamercolor*{author}{fg=black}
\setbeamercolor*{institute}{fg=black}
\setbeamercolor*{date}{fg=black}
\usecolortheme[named=black]{structure}
\setbeamercolor{alerted text}{fg=black}
\setbeamercolor{block title}{fg=black}
\setbeamercolor{block title example}{fg=black!90!black}
\setbeamercolor{normal text}{fg=black}
%\useoutertheme{shadow}

\begin{document}

\frame{\titlepage}

\section{Contexte}

\begin{frame}{Centrale Nantes et le Libre}
    \begin{block}{Collaborations}
	\begin{itemize}[<+->]
	    \item Markus
	    \item OrbisGis
	    \item OpenOffice.org OpenOffice.org4Kids
	\end{itemize}
    \end{block}
\end{frame}

\begin{frame}{OOo/OOo4Kids à Centrale Nantes}
    \begin{block}{Concours HP - 21 Tablet PCs gagnés en 2008}
	\begin{itemize}[<+->]
	    \item Cartable électronique libre
            \item GNU/Linux
            \item Amélioration d'OpenOffice.org Impress pour les Tablet-PCs
	\end{itemize}
    \end{block}
\end{frame}

\begin{frame}{OOo/OOoKids à Centrale Nantes}
    \begin{block}{Module d'annotation dans OpenOffice}
	\begin{itemize}
	    \item Codé en C++
	    \item Possibilité de changer
	    \begin{itemize}
		\item taille
		\item couleur
	    \end{itemize}
	\end{itemize}
    \end{block}
\end{frame}

\begin{frame}{OOo/OOo4Kids à Centrale Nantes}
    \begin{block}{OOo4Kids}
	\begin{itemize}[<+->]
	    \item Logiciel de bureautique libre et gratuit pour les 7-12 ans
	    \item OpenOffice.org simplifié
	    \item Adapté aux programmes d'enseignement.
	\end{itemize}
    \end{block}
\end{frame}

\begin{frame}{Module d'annotations}
    \begin{block}{2009}
        \begin{itemize}[<+->]
            \item Gomme
	    \item Sauvegarde des annotations
        \end{itemize}
    \end{block}
\end{frame}

\begin{frame}{Module d'annotations}
    \begin{block}{2009}
        \begin{itemize}[<+->]
            \item Debogage des patchs des années précédentes
	    \item Switch entre gomme et crayon
        \end{itemize}
    \end{block}
\end{frame}

\begin{frame}{Module d'annotations}
    \begin{block}{2011: Objectifs}
        \begin{itemize}[<+->]
            \item Mode curseur
            \item Extensibilité
        \end{itemize}
    \end{block}
\end{frame}

\section{Cadre Technique}

\begin{frame}{Travail préliminaire}
    \begin{block}{Documentation}
        \begin{itemize}[<+->]
            \item Rapports des années précédentes
            \item Wiki
        \end{itemize}
    \end{block}
\end{frame}

\begin{frame}{Travail préliminaire}
    \begin{block}{Cahier des charges}
        \begin{itemize}[<+->]
            \item Diagrammes d'état
            \item 
        \end{itemize}
    \end{block}
\end{frame}

\begin{frame}{Travail préliminaire}
    \begin{block}{Environnement de développement}
        \begin{itemize}[<+->]
            \item Compilation (dmake, ccache, distcc)
            \item Versionnement (SVN -> Git)
        \end{itemize}
    \end{block}
\end{frame}

\end{document}
