\documentclass[hyperref,french,usenames,xcolor=dvipsnames]{beamer}
 \mode<presentation>
{
\usepackage{beamerthemesplit}
\usetheme[compress,secheader]{Madrid}
%  \usecolortheme{orchid}
%	\setbeamercolor{alerted text}{fg=red!65!black}
  \setbeamercovered{transparent}
}
\usepackage{amsthm}
\usepackage{amsfonts}
\usepackage{amsmath}
\usepackage{graphicx}
%\usepackage{epsfig}
\usepackage{xspace}
\usepackage{stmaryrd}
%\usepackage{soul}
\usepackage[utf8]{inputenc}
%\usepackage[french]{algorithme}
%\usepackage[T1]{fontenc} %pour un beau PDF sous Linux ; à retirer sous Mac
\usepackage[french]{babel}

%\usepackage{ulem}
%\usepackage[english]{babel}
%\usepackage{times}
%\usepackage[utf8]{inputenc}
%\usepackage{amssymb}
%\usepackage{movie15}
%\usepackage{graphicx,color}
%\usepackage{hyperref}


%-----------------------------------------------------------

\definecolor{newcolor}{rgb}{0, 0, .90}
\definecolor{impcolor}{rgb}{.90, 0, 0}
\definecolor{darkergreen}{rgb}{0,0.5,0}
\definecolor{myorange}{rgb}{0.8,0.7,0}
\definecolor{myviolet}{rgb}{0.7,0.0,0.7}

\definecolor{lightpurple}{rgb}{0.83,0.27,1}
\definecolor{lightblue}{rgb}{0.27,0.9,0.9}



%\newcommand{\texthl}[1]{{\color{blue}#1}}
%\newcommand{\jaune}[1]{{\color{blue}#1}}
%\newcommand{\texthlb}[1]{{\color{orange}#1}}
%\newcommand{\orange}[1]{{\color{orange}#1}}
%\newcommand{\verte}[1]{{\color{green}#1}}
%\newcommand{\trad}{{\color{orange}{\bf $\leadsto$}}}

%\newcommand{\textcite}[1]{{\color{lightpurple}[#1]}}
%\newcommand{\textciten}[1]{{\color{lightpurple}#1}}

\newcommand{\texthl}[1]{{\color{red}#1}}
\newcommand{\jaune}[1]{{\color{red}#1}}
\newcommand{\texthlb}[1]{{\color{orange}#1}}
\newcommand{\orange}[1]{{\color{orange}#1}}
\newcommand{\verte}[1]{{\color{green}#1}}
\newcommand{\trad}{{\color{orange}{\bf $\leadsto$}}}

\newcommand{\textcite}[1]{{\color{myviolet}[#1]}}
\newcommand{\textciten}[1]{{\color{myviolet}#1}}

\newcommand{\montilde}{$\sim$}


\title[OOo Edu]%
{Le cercle vertueux de la participation d'étudiants à des projets libres}

\author[C. \textsc{Delafargue}, M. \textsc{Magnin}, N. \textsc{Varoquaux}, B. \textsc{Vialle}]%
{Clément \textsc{Delafargue}, Morgan \textsc{Magnin}, Nelle \textsc{Varoquaux} et Benjamin \textsc{Vialle}
}
\institute[ECN]{
\structure{
École Centrale de Nantes}
}

\date[12/07/2011]{Rencontres Mondiales du Logiciel Libre - 12/07/11}

%\date[] % (optional)
%{}

\subject{Rencontres Mondiales du Logiciel Libre - 12/07/11}

\AtBeginSection[] % Do nothing for \section*
{
\frame<beamer>
	{
	\frametitle{Sommaire}
	\tableofcontents[current]
	}
}

\begin{document}

\frame{\titlepage}

% La proposition est d'axer la présentation du plus pédagogique au plus technique
% Faisant ainsi le lien avec la seconde présentation, orientée "technique"/"contribution"

\section{Introduction}

\frame
{
  \frametitle{École Centrale de Nantes}

  \begin{block}{École d'ingénieur généraliste}
  Accessible principalement après les classes préparatoires, elle développe :
    \begin{itemize}
      \item des compétences scientifiques et techniques
      \item des compétences humaines :
      \begin{itemize}
                \item une capacité à {\bf s'intégrer}
                \item une capacité à {\bf communiquer}
                \item une capacité à {\bf partager}
      \end{itemize}
    \end{itemize}
  \end{block}
  
  \begin{block}{Enseignement}
        Deux ans de tronc commun, suivi d'une année de spécialisation
  \end{block}
  
  \begin{alertblock}{}
    Participation d'étudiants de troisième année option informatique à des projets libres, pour ceux qui le souhaitent.
  \end{alertblock}
}

\frame
{
  \frametitle{Motivations}

\begin{alertblock}{Objectif}
Comment aider les étudiants à acquérir les \textbf{compétences} nécessaires pour le milieu professionnel et \textbf{évaluer} \textbf{efficacement} les travaux rendus par les étudiants en TP/Projet ?
\end{alertblock}

\begin{block}{Pédagogie par projet}
Autonomie et initiative
\end{block}

}

\frame
{
  \frametitle{Motivations}

\begin{alertblock}{Objectif}
Comment susciter l'\textbf{adhésion} des étudiants ? 
\end{alertblock}

\begin{block}{Faire sens}
Des projets répondants à des \textbf{besoins} ressentis par les élèves et/ou \textbf{valorisants}
\end{block}
}

\frame
{
  \frametitle{Motivations}

\begin{alertblock}{Objectif}
Comment faire bénéficier la \textbf{communauté} du travail des étudiants réalisés en projet ?
\end{alertblock}

\begin{block}{Philosophie libre}
Différents projets, notamment une collaboration autour de OpenOffice.org et OOo4Kids depuis 2008
\end{block}
}

%À développer... 
\section{Un peu d'histoire}

\frame
{
  \frametitle{Le logiciel libre à Centrale Nantes}

\begin{block}{Sensibiliser les étudiants à une approche \textbf{multi-culturelle}}
\begin{itemize}
\item Double-boot Windows/Linux 
\item Recommandation d'usages : Firefox, OpenOffice.org, …
\end{itemize}
\end{block}

\begin{block}{Apprentissage par projets}
\begin{itemize}
\item Développement autour de logiciels dédiés à la recherche 
\item Logiciel MarkUs - cf. autres exposés RMLL 
\item OOo/OOo4Kids
\end{itemize}
\end{block}

}


\begin{frame}{OOo/OOo4Kids à Centrale Nantes}
    \begin{block}{OOo4Kids}
	\begin{itemize}
	    \item Logiciel de bureautique libre et gratuit pour les 7-12 ans
	    \item OpenOffice.org simplifié (\textit{Impress})
	    \item Adapté aux programmes d'enseignement
	\end{itemize}
    \end{block}
\end{frame}

\begin{frame}{Module d'annotations}
    \begin{block}{2009}
        \begin{itemize}
            \item Gomme
	    \item Sauvegarde des annotations
        \end{itemize}
    \end{block}
    \begin{block}{2010}
        \begin{itemize}
            \item Debogage des patchs des années précédentes
	        \item Switch entre gomme et crayon
        \end{itemize}
    \end{block}
    \begin{block}{2011}
        \begin{itemize}
            \item Mode curseur
        \end{itemize}
    \end{block}
\end{frame}

\section{Recommandations}

\frame
{
  \frametitle{S'assurer de l'intérêt de l'équipe de développement}

\begin{alertblock}{Objectif}
Garantir l'\textbf{intégration} des étudiants dans l'équipe 
\end{alertblock}

\begin{block}{Disposer d'une structure d'accueil}
\begin{itemize}
\item Projet Education de OpenOffice.org 
\item Association Educooo
\end{itemize}
\end{block}

\begin{block}{Vérifier l'intérêt pour la fonctionnalité envisagée}
\begin{itemize}
\item Besoins ressentis par la communauté des utilisateurs 
\item Tickets restés ouverts  
\end{itemize}
\end{block}

}

\frame
{
  \frametitle{Viser des étudiants intéressés}

\begin{alertblock}{Objectif}
Maximiser les chances de réussite du projet 
\end{alertblock}

\begin{block}{Identifier des élèves motivés}
\begin{itemize}
\item Des projets plus \textbf{complexes} que d'autres…
\item Mais également plus \textbf{professionnalisants}
\item Miser sur une certaine continuité
\end{itemize}
\end{block}

}

\frame
{
  \frametitle{Assurer un bon encadrement}

\begin{alertblock}{Objectif}
Assurer la présence d'un mentor technique
\end{alertblock}

\begin{block}{La nécessité d'un mentor technique}
\begin{itemize}
\item Lien avec communauté
\item Réponse aux questions techniques
\item Continuité avec les projets précédents
\item Bonnes pratiques
\item Participe à l'évaluation
\end{itemize}
\end{block}

}

\frame
{
  \frametitle{Valider l'apport pédagogique du projet}

\begin{alertblock}{Objectif}
S'assurer que les \textbf{compétences attendues} sont bien développées au cours du projet
\end{alertblock}

\begin{block}{Garantir l'implication du tuteur pédagogique}
\begin{itemize}
\item Repère au sein de l'institution 
\item Interface entre les communautés 
\item Reconnaissance académique du travail accompli par les élèves
\end{itemize}
\end{block}

}

\frame
{
  \frametitle{Communiquer}

\begin{alertblock}{Objectif}
Assurer la \textbf{cohésion} des communautés en jeu 
\end{alertblock}

\begin{block}{Pendant le projet}
\begin{itemize}
\item Réunion fréquentes, tant en présentiel qu'à distance 
\item Blogs, wikis, réseaux sociaux
\end{itemize}
\end{block}

\begin{block}{Après le projet}
\begin{itemize}
\item Communication institutionelle 
\item Colloques 
\end{itemize}
\end{block}

}

\frame
{
  \frametitle{Avoir conscience des obstacles}

\begin{alertblock}{Objectif}
Surmonter les difficultés propres à ce type de projet 
\end{alertblock}

\begin{block}{Complexité}
Quelques lignes de code peuvent représenter un travail colossal 
\end{block}

\begin{block}{Intégration}
\begin{itemize}
\item Enjeu crucial 
\item Rarement terminée à la fin du projet
\item Doit être portée par une personne dûment identifiée
\end{itemize}
\end{block}

\begin{block}{Continuité}
\begin{itemize}
\item Documentation (commentaires du code, rapport, etc.) 
\item Interactions avec d'anciens étudiants impliqués 
\end{itemize}
\end{block}

}

\frame
{
  \frametitle{Synthèse}

\begin{block}{Un bilan positif}
\begin{itemize}
\item 6 projets, 3 années universitaires, 16 étudiants impactés
\item \textbf{Continuité} des projets 
\item Une contribution palpable
\item \textbf{Professionnalisation} accrue
\item Une sensibilisation aux enjeux du libre
\end{itemize}

\end{block}

}

\frame
{
  \frametitle{Perspectives}

\begin{block}{À suivre}
\begin{itemize}
\item Perfectionner les fonctionnalités existantes
\item Défricher d'autres besoins 
\item Consolider la création de cercles vertueux utilisateur $\rightarrow$ contributeur $\rightarrow$ mentor
\end{itemize}

\end{block}

}

\frame{
  \frametitle{Plus d'informations}
  
\begin{block}{Liens et contacts}
\begin{itemize}
\item OpenOffice.org et OOo4Kids : \url{http://wiki.ooo4kids.org/index.php/Applications/ CentraleNantes}
\item Blog EAT-TICE de l'Ecole Centrale de Nantes : \url{http://eat-tice.ec-nantes.fr}
\end{itemize}
\end{block}
}


\end{document}
