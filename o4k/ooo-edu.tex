\documentclass[hyperref,french,usenames,xcolor=dvipsnames]{beamer}
 \mode<presentation>
{
\usepackage{beamerthemesplit}
\usetheme[compress,secheader]{Madrid}
%  \usecolortheme{orchid}
%	\setbeamercolor{alerted text}{fg=red!65!black}
  \setbeamercovered{transparent}
}
\usepackage{amsthm}
\usepackage{amsfonts}
\usepackage{amsmath}
\usepackage{graphicx}
%\usepackage{epsfig}
\usepackage{xspace}
\usepackage{stmaryrd}
%\usepackage{soul}
\usepackage[utf8]{inputenc}
%\usepackage[french]{algorithme}
%\usepackage[T1]{fontenc} %pour un beau PDF sous Linux ; à retirer sous Mac
\usepackage[french]{babel}

%\usepackage{ulem}
%\usepackage[english]{babel}
%\usepackage{times}
%\usepackage[utf8]{inputenc}
%\usepackage{amssymb}
%\usepackage{movie15}
%\usepackage{graphicx,color}
%\usepackage{hyperref}


%-----------------------------------------------------------

\definecolor{newcolor}{rgb}{0, 0, .90}
\definecolor{impcolor}{rgb}{.90, 0, 0}
\definecolor{darkergreen}{rgb}{0,0.5,0}
\definecolor{myorange}{rgb}{0.8,0.7,0}
\definecolor{myviolet}{rgb}{0.7,0.0,0.7}

\definecolor{lightpurple}{rgb}{0.83,0.27,1}
\definecolor{lightblue}{rgb}{0.27,0.9,0.9}



%\newcommand{\texthl}[1]{{\color{blue}#1}}
%\newcommand{\jaune}[1]{{\color{blue}#1}}
%\newcommand{\texthlb}[1]{{\color{orange}#1}}
%\newcommand{\orange}[1]{{\color{orange}#1}}
%\newcommand{\verte}[1]{{\color{green}#1}}
%\newcommand{\trad}{{\color{orange}{\bf $\leadsto$}}}

%\newcommand{\textcite}[1]{{\color{lightpurple}[#1]}}
%\newcommand{\textciten}[1]{{\color{lightpurple}#1}}

\newcommand{\texthl}[1]{{\color{red}#1}}
\newcommand{\jaune}[1]{{\color{red}#1}}
\newcommand{\texthlb}[1]{{\color{orange}#1}}
\newcommand{\orange}[1]{{\color{orange}#1}}
\newcommand{\verte}[1]{{\color{green}#1}}
\newcommand{\trad}{{\color{orange}{\bf $\leadsto$}}}

\newcommand{\textcite}[1]{{\color{myviolet}[#1]}}
\newcommand{\textciten}[1]{{\color{myviolet}#1}}

\newcommand{\montilde}{$\sim$}


\title[OOo Edu]%
{Le cercle vertueux de la participation d'étudiants à des projets libres}

\author[C. \textsc{Delafargue}, M. \textsc{Magnin}, N. \textsc{Varoquaux}, B. \textsc{Vialle}]%
{Clément \textsc{Delafargue}, Morgan \textsc{Magnin}, Nelle \textsc{Varoquaux} et Benjamin \textsc{Vialle}
}
\institute[ECN]{
\structure{
École Centrale de Nantes}
}

\date[12/07/2011]{Rencontres Mondiales du Logiciel Libre - 12/07/11}

%\date[] % (optional)
%{}

\subject{Rencontres Mondiales du Logiciel Libre - 12/07/11}

\AtBeginSection[] % Do nothing for \section*
{
\frame<beamer>
	{
	\frametitle{Sommaire}
	\tableofcontents[current]
	}
}

\begin{document}

\frame{\titlepage}

% La proposition est d'axer la présentation du plus pédagogique au plus technique
% Faisant ainsi le lien avec la seconde présentation, orientée "technique"/"contribution"

\section{Introduction}

\frame
{
  \frametitle{Motivations}

\begin{alertblock}{Objectif}
Comment aider les étudiants à acquérir les \textbf{compétences} nécessaires pour le milieu professionnel et \textbf{évaluer} \textbf{efficacement} les travaux rendus par les étudiants en TP/Projet ?
\end{alertblock}

\begin{block}{Pédagogie par projet}
Autonomie et initiative
\end{block}

\begin{alertblock}{Objectif}
Comment susciter l'\textbf{adhésion} des étudiants ? 
\end{alertblock}

\begin{block}{Faire sens}
Des projets répondants à des \textbf{besoins} ressentis par les élèves et/ou \textbf{valorisants}
\end{block}
}

\frame
{
  \frametitle{Motivations}

\begin{alertblock}{Objectif}
Comment faire bénéficier la \textbf{communauté} du travail des étudiants réalisés en projet ?
\end{alertblock}

\begin{block}{Philosophie libre}
Différents projets, notamment une collaboration autour de OpenOffice.org et OOo4Kids depuis 2008
\end{block}
}

%À développer... 

\end{document}