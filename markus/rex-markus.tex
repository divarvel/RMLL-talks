\documentclass[hyperref,french,usenames,xcolor=dvipsnames]{beamer}
 \mode<presentation>
{
\usepackage{beamerthemesplit}
\usetheme[compress,secheader]{Madrid}
%  \usecolortheme{orchid}
%	\setbeamercolor{alerted text}{fg=red!65!black}
  \setbeamercovered{transparent}
}

\usepackage{amsthm}
\usepackage{amsfonts}
\usepackage{amsmath}
\usepackage{graphicx}
%\usepackage{epsfig}
\usepackage{xspace}
\usepackage{stmaryrd}
%\usepackage{soul}
\usepackage[utf8]{inputenc}
%\usepackage[french]{algorithme}
%\usepackage[T1]{fontenc} %pour un beau PDF sous Linux ; à retirer sous Mac
\usepackage[french]{babel}

%\usepackage{ulem}
%\usepackage[english]{babel}
%\usepackage{times}
%\usepackage[utf8]{inputenc}
%\usepackage{amssymb}
%\usepackage{movie15}
%\usepackage{graphicx,color}
%\usepackage{hyperref}


%-----------------------------------------------------------

\definecolor{newcolor}{rgb}{0, 0, .90}
\definecolor{impcolor}{rgb}{.90, 0, 0}
\definecolor{darkergreen}{rgb}{0,0.5,0}
\definecolor{myorange}{rgb}{0.8,0.7,0}
\definecolor{myviolet}{rgb}{0.7,0.0,0.7}

\definecolor{lightpurple}{rgb}{0.83,0.27,1}
\definecolor{lightblue}{rgb}{0.27,0.9,0.9}



%\newcommand{\texthl}[1]{{\color{blue}#1}}
%\newcommand{\jaune}[1]{{\color{blue}#1}}
%\newcommand{\texthlb}[1]{{\color{orange}#1}}
%\newcommand{\orange}[1]{{\color{orange}#1}}
%\newcommand{\verte}[1]{{\color{green}#1}}
%\newcommand{\trad}{{\color{orange}{\bf $\leadsto$}}}

%\newcommand{\textcite}[1]{{\color{lightpurple}[#1]}}
%\newcommand{\textciten}[1]{{\color{lightpurple}#1}}

\newcommand{\texthl}[1]{{\color{red}#1}}
\newcommand{\jaune}[1]{{\color{red}#1}}
\newcommand{\texthlb}[1]{{\color{orange}#1}}
\newcommand{\orange}[1]{{\color{orange}#1}}
\newcommand{\verte}[1]{{\color{green}#1}}
\newcommand{\trad}{{\color{orange}{\bf $\leadsto$}}}

\newcommand{\textcite}[1]{{\color{myviolet}[#1]}}
\newcommand{\textciten}[1]{{\color{myviolet}#1}}

\newcommand{\montilde}{$\sim$}


\def \PH {\mathcal{PH}}
\def \PHb {\overline{\PH}}
\def \Hits {\mathcal{H}}
\def \GRN {\mathcal{G}}
\newcommand{\hitp}[2]{\mbox{$#1\rightarrow#2$}}
\newcommand{\hitb}[2]{\mbox{$#1\Rsh#2$}}
\newcommand{\hits}[3]{\mbox{$#1\xrightarrow{#3}#2$}}
\newcommand{\hitpath}[4]{\mbox{$#2\xrightarrow{#1}^*#3\Rsh#4$}}
\newcommand{\hit}[3]{\mbox{$#1\rightarrow#2\Rsh#3$}}

\title[REX MarkUs]%
{Retour d’expérience sur le déploiement à Centrale Nantes d’une application web d’annotation du code des étudiants : MarkUs}

%\subtitle{Applications aux réseaux de Petri temporels et aux automates
%  temporisés}

\author[M. \textsc{Magnin}, G. \textsc{Moreau}, N. \textsc{Varoquaux}, B. \textsc{Vialle}]%
{Morgan \textsc{Magnin}, Guillaume \textsc{Moreau}, Nelle \textsc{Varoquaux} et Benjamin \textsc{Vialle}
}
\institute[ECN]{
\structure{
École Centrale de Nantes}
}

\date[14/07/2011]{Rencontres Mondiales du Logiciel Libre - 14/07/11}

%\date[] % (optional)
%{}

\subject{Rencontres Mondiales du Logiciel Libre - 14/07/11}

\AtBeginSection[] % Do nothing for \section*
{
\frame<beamer>
	{
	\frametitle{Sommaire}
	\tableofcontents[current]
	}
}

\begin{document}

\frame{\titlepage}

% La proposition est d'axer la présentation du plus pédagogique au plus technique
% Faisant ainsi le lien avec la seconde présentation, orientée "technique"/"contribution"

\section*{Introduction}

\frame
{
  \frametitle{Des besoins identifiés}

\begin{alertblock}{Motivation}
Comment gérer et évaluer efficacement les travaux rendus par les étudiants en TP/Projet ?
\end{alertblock}

\begin{block}{Usage de MarkUs}
\begin{itemize}
\item Déployé à Nantes depuis septembre 2010
\item Participation au développement depuis l'été 2009
\item Terrains d'utilisation
\begin{itemize}
\item Enseignements d'informatique (rapport et code)
\item Promotions de plus de 350 étudiants
\item 20 enseignants impactés
\end{itemize}
\end{itemize}
\end{block}
}

\section*{Contexte}

\subsection*{Motivation}

\frame
{
  \frametitle{Du côté des enseignants}

\begin{itemize}
\item Gros volume de soumissions à traiter (plusieurs centaines par TP)
\item Difficulté d'harmonisation des facteurs de correction d'un chargé de TD/TP à l'autre
\item Gestion papier
\begin{itemize}
\item Amoncellement de piles
\item Retour des dossiers aux étudiants
\end{itemize}
\item Gestion par courriels
\begin{itemize}
\item Erreurs dans le destinataire
\item Archives .zip illisibles
\item Lourdeurs
\end{itemize}
\end{itemize}
}

\frame
{
  \frametitle{Du côté des étudiants}

\begin{itemize}
\item Difficulté pour récupérer/consulter ses travaux corrigés
\item Gestion papier
\begin{itemize}
\item Perte de rapports
\item Partage de la copie avec son binôme ?
\end{itemize}
\item Gestion par courriels
\begin{itemize}
\item Erreurs dans le destinataire
\item Un courriel parmi d'autres
\end{itemize}
\end{itemize}
}

\subsection*{Centrale Nantes et le logiciel libre}

\frame
{
  \frametitle{Un intérêt de longue date}

\begin{itemize}
\item En termes d'utilisation
\begin{itemize}
\item Incitation à utiliser GNU/Linux
\item Promotion de OpenOffice.org, Firefox, etc.
\end{itemize}
\item En termes de développement
\begin{itemize}
\item Logiciels pour la recherche
\item Collaboration autour de OpenOffice.org (depuis 2008)
\end{itemize}
\end{itemize}
}

\subsection*{MarkUs}

% Quelques mots de présentation historique autour de MarkUs

\section*{Impact sur l'enseignement et l'apprentissage}

% Cette partie pour donner un aperçu des fonctionnalités de MarkUs et leur impact sur l'enseignement

\subsection*{Avantages pour les enseignants}

\subsection*{Avantages pour les étudiants}

\subsection*{Démonstration}

\section*{Environnement technique}

\subsection*{Déploiement à Centrale Nantes}

\subsection*{Contributions passées et à venir}

\section*{Conclusion}

\subsection*{Bilan intermédiaire}

\frame
{
  \frametitle{Synthèse}

\begin{itemize}
\item Constitution de cercles vertueux
\begin{itemize}
\item Incitation à utiliser GNU/Linux
\item Promotion de OpenOffice.org, Firefox, etc.
\end{itemize}
\item En termes de développement
\begin{itemize}
\item Logiciels pour la recherche
\item Collaboration autour de OpenOffice.org (depuis 2008)
\end{itemize}
\end{itemize}
}

\subsection*{Perspectives}

\subsection*{Contacts}

%\bibliography{biblio}
%\bibliographystyle{alpha}

\end{document}